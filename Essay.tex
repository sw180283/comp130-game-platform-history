\documentclass{scrartcl}

\usepackage[hidelinks]{hyperref}
\usepackage[none]{hyphenat}
\usepackage{setspace}
\doublespace

\title{[Was The Contribution Of The SG-1000 Console Significant Given The Historical, Sociological and Technological Context Surrounding Release]}
\subtitle{COMP130 - Game Platform History Essay}
\date{2015-11-02}
\author{SW180283}

\begin{document}
	\pagenumbering{gobble}
	\maketitle
	\abstract{This is an introductory essay into the contextual events surrounding the SEGA SG-1000 game console. The focus of the essay is on the contributory factors each event  and console release had on the production and later development of SEGA's first venture into the home video game industry.}
	\newpage
	\pagenumbering{arabic}
	\tableofcontents
	\newpage
	
	\section{Introduction}

In this essay, I shall be exploring the motives, execution and competitive response of the SEGA SG-1000 home video game console. I shall conclude my findings after determining the contributory factors that lead to the design of the SG-1000 platform in terms of hardware, functionality and design, with the specific question of concurrence to whether the SG-1000 enhanced the third generation and subsequent development of game consoles. In addition, the discussion of whether the SG-1000 lacked ingenuity in terms of hardware design with comparative relevance to both second generation and third generation game consoles. Being able to identify the significance of the contribution that this first home console version SEGA made at such a crucial moment in game history, is a reference to a historical landmark for many companies and businesses.

	\section{Main Body}
	\subsection{Following The Second Generation}
	
During the second generation of game platforms, the market was heavily dominated by the Atari Corporation; supported by a revenue increase “tenfold from \$39 to \$415 million between 1976 and 1980” \cite{Sutton1986}. While its competitors still maintained a profit, they were the most popular with their selection of games such as: \cite{game:MissionControl}, \cite{game:SpaceInvaders} and \cite{game:Asteroids}.

This evidentially supported SEGA’s motives, showing that the game console industry could be very successful and profitable. However, many other companies also noticed this opportunity which later lead to the crash of 1983.
The test market for the SG-1000 was noted to be 1981, suggesting that SEGA’s initial intentions were to develop a competitive product that would also return similar profits as the Atari 2600 \cite{SegaRetro}.
A problem with this test market was that the SG-1000 attempted to release a console with hardware and features expected of a second generation console, which meant that it is arguable as to if it should be known as second or third generation console.
Considering the release of the NES, it would appear that the only reason for it to be classed as a third generation console would be that the SG-1000 was released on the same day, and the NES was clearly in a new stage of game console development.

In 1983, the industry of home computer video game systems was a confusing and over saturated market due to the abundance of companies attempting to make profit on the apparent trend, some of which were not typically associated with the game, or even the entertainment, industry such as Quaker Oaks \cite{Wolf2012}. This lead to a multitude of unregulated games in terms of quality and their censorship of adult themes. Due to the game market dilution and media portrayal, the reputation of video game consoles lead to both consumers and distributors feeling doubtful for the industries future \cite{Kent2010Chapter17}.
	
	\subsection{The Release}

On July 15, 1983, the SG-1000 was first released in Japan for ¥15,000 (\$125) and later distributed to other areas of Asia. While it did reach countries such as Australia and some parts of Europe, it would not be until later revisions of the console were made for SEGA’s home console to reach the North American audience \cite{Pettus2013}.

While SEGA had a strong footing in the public eye as an arcade successful business, the SG-1000 was their first attempt at creating a console that was expected to run some of the same games with similar capabilities as their arcade machines, but on a much smaller platform. This is evidenced by use of off-the-shelves components, identical to those used in the Colecovision: Zilog Z80 A 3.58-MHz CPU, Texas Instruments TMS9928A 16-color video processor and Texas Instruments SN76489 sound chip \cite{VideoGameConsoleLibrary}.

Due to their inexperience, this hardware compromise due a lack of physical space, made SEGA’s choice to use common components.Therefore, there has always been a comparison drawn between the two and often is referred to as the reason for the SG-1000’s restrictive contribution to the industry.

With a console that was not that much different in terms of technical specifications than its predecessors, and with similar designs to a controller that looked like the Atari 2600 and Colecovision combined, did not give much in terms of innovativity. From the initial look of the SG-1000 in terms of design, it appeared ordinary and an imitation console repackaged \cite{SegaSG-1000}. 

This also meant that when the Nintendo Entertainment System was released, their new processors provided higher quality audio, graphical and sound output that embarrassed the SG-1000 substandard abilities. However, there was a recall of the console due to consumers complaining of freezing after playing certain games. This lead to a replacement of the motherboard and a possible opening for SEGA to steal the market. However this move increased Nintendo’s popularity due to their vigilance to fix the problem \cite{Loguidice2014}.
	
	\subsection{Future Of SEGA}

The future of the SG-1000 was influenced by the consumer feedback, including the main issue of the unresponsive hardwired controller and substandard graphics. While hardware controllers were a common feature of many consoles, the comparison and use of both the hardwired and detachable controller on the same console made the difference apparent \cite{VideoGameConsoleLibrary}.

The comments of the uneconomical design of the controllers, with the placement of the buttons on the side, also influenced the later design of the Mark II which features a similar controller design as the original NES controller; with a d-pad and two top facing buttons \cite{ChrisKohler2009}. When Nintendo released their console in North America, SEGA was fuelled to also infiltrate the US market with their newly released SEGA Master System. 

This is evident that the business approach and critical consumer response to both Nintendo and SEGA systems influenced the later production and entrepreneurial approach to design, marketing and target audience \cite{Kent2010TheBattle}.

	\section{Conclusion}

While some may argue that SEGA was successful despite the SG-1000 contribution; it is evidenced that by making mistakes and receiving consumer feedback that SEGA’s success was not attributed but perhaps improved through this console. The comments and fan based support motivated later products, such as hardware and controller design.

While it appears that the SG-1000 developed no innovative features or display of hardware ingenuity, it was the start of the SEGA’s venture into the games industry of home consoles.

The competition between SEGA and Nintendo fuelled its production and later releases of games and their systems, such as the Sega Master System, Genesis and the Dreamcast. It was a battle that appeared to influence both parties for the better; each attempting to produce more entertaining features and games that otherwise would not be designed due to lack of competition.

Therefore, while the SG-1000 on the surface appeared to have been a failure in its defeat against Nintendo, it was a pivotal point in game history that contributed to one of the leading developers in game console production and inspiring a generation.

	
	\newpage
	
	\bibliographystyle{plain}
	\bibliography{bibReferences1}
			
\end{document}
